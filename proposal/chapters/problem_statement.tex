\vspace*{5mm}

\paragraph*{} {
	A pipeline architecture is a system in which multiple tasks are divided into subtasks. These subtasks are processed in different stages in parallel where the output of one stage serves as the input of the next stage. While one stage is working on some input, the consecutive stages are working on the output produced by the previous stages thereby increasing the throughput of the system without affecting latency. The most desirable goal of a pipeline architecture is that every stage is balanced i.e. every stage takes roughly the same amount of time to compute on the input given to it.
}

\paragraph*{} {
	In traditional distributed systems, a server communicates with one or more clients whenever they are involved in processing some part of a task. When one client finishes its work, it sends the information to server and the server then gives that information to the other client. We reduce this interaction by introducing a pipeline such that the clients mimic the stages of the pipeline and the output produced by one client/stage is directly passed to the next client/stage. The previous client can now receive the next set of input and the process goes on. 
}

\paragraph*{} {
	Through this project, we aim to improve the throughput of a distributed system by integrating a pipeline architecture with a distributed system. 
}